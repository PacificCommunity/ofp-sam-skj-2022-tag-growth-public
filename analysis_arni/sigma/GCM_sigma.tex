\documentclass[fleqn]{article}
\usepackage{datetime}\newdateformat{mydate}{\THEDAY~\monthname~\THEYEAR}\mydate
\usepackage[a4paper,height=23cm]{geometry}
\usepackage{parskip}
\usepackage{color}
\usepackage[hidelinks,linktoc=all,pdfpagemode=None]{hyperref}
\begin{document}

\title{Variance treatment in the GCM model}
\author{Arni Magnusson}
\maketitle

\section*{Likelihood}

In the GCM model (\textcolor[rgb]{0,0.1,0.5}%
{\href{https://doi.org/10.1007/s00227-018-3336-9}{Maunder et al. 2018}}),
$\sigma$ is used in the calculation of log-likelihoods in a traditional manner:

\begin{verbatim}
  dnorm(y, mu, sigma, true)
\end{verbatim}

\section*{Unequal variances}

In the calculation of $\sigma$, it increases with the size of fish:

\begin{displaymath}
  \sigma_i \;=\; \sigma_a \;+\; (\sigma_b + \sigma_\mathrm{ME})\times \hat L_i
\end{displaymath}

\section*{Model settings}

In the default model settings, some $\sigma$ coefficients are fixed and others
estimated:

\begin{itemize}
  \item[] $\sigma_a$ is fixed near zero ($10^{-7}$)
  \item[] $\sigma_b$ is estimated
  \item[] $\sigma_\mathrm{ME}$ is fixed at 0.0222
\end{itemize}

\section*{Source code}

\begin{verbatim}
  PARAMETER(ln_sd_a);
  PARAMETER(ln_sd_b);
  PARAMETER(ln_sd_MEb);

  sd_a = exp(ln_sd_a);
  sd_b = exp(ln_sd_b);
  sd_MEb = exp(ln_sd_MEb);

  Lrel_sd(i) = sd_a + sd_b * Lrel_pred(i);
  Lrel_MEsd(i) = sd_MEb * Lrel_pred(i);

  Lrel_nll(i) = -dnorm(Lrel_obs(i),
                       Lrel_pred(i),
                       Lrel_sd(i) + Lrel_MEsd(i),
                       true);
\end{verbatim}

\end{document}
