\documentclass[fleqn]{article}
\usepackage{datetime}\newdateformat{mydate}{\THEDAY~\monthname~\THEYEAR}\mydate
\usepackage[a4paper,height=23cm]{geometry}
\usepackage{parskip}
\usepackage{color}
\usepackage[hidelinks,linktoc=all,pdfpagemode=None]{hyperref}
\begin{document}

\title{Description of the VBtag model}
\author{Arni Magnusson}
\date{8 July 2022}
\maketitle

\section*{Growth}

Growth follows a traditional von Bertalanffy form:

\begin{displaymath}
  \hat L_i \;=\; L_\infty \Big(1-e^{-k(t_i-t_0)}\Big)
\end{displaymath}

\section*{Likelihood}

A traditional normal likelihood is used:

\begin{verbatim}
  dnorm(y, mu, sigma)
\end{verbatim}

\vspace{1ex}

The model has three likelihood components, baesd on the fit to observed lengths
at tag release, lengths at tag recaptures, and lengths from the otolith data:

\begin{displaymath}
  f \;=\; \log L_\mathrm{rel}
  \;+\; \log L_\mathrm{rec}
  \;+\; \log L_\mathrm{oto}
\end{displaymath}

\section*{Unequal variances}

The variability in length at age ($\sigma$) varies with age:

\begin{displaymath}
  \log\sigma_i \;=\; \mathrm{intercept} \;+\; \mathrm{slope} \times t_i
\end{displaymath}

\section*{Model parameters}

\begin{tabular}{ll}
  $L_\infty$, $k$, $t_0$ & growth curve coefficients\\[1ex]
  $\sigma_a$, $\sigma_b$ & sd(length) at ages $a$ and $b$\\[1ex]
  age & vector of estimated age at release for all tagged fish\\[1ex]
\end{tabular}

Parameters are estimating using traditional MLE, except the sigma parameters are
fixed (estimated iteratively and externally).

\section*{Input data}

\begin{verbatim}
  DATA_VECTOR(Lrel);        // length at release (tags)
  DATA_VECTOR(Lrec);        // length at recapture (tags)
  DATA_VECTOR(liberty);     // time at liberty (tags)
  DATA_VECTOR(Aoto);        // age (otoliths)
  DATA_VECTOR(Loto);        // length (otoliths)
  DATA_SCALAR(a);           // younger age where sd(length) is sigma_a
  DATA_SCALAR(b);           // older age where sd(length) is sigma_b
\end{verbatim}

\section*{Source code}

\textcolor[rgb]{0,0.1,0.5}%
{\href{https://github.com/PacificCommunity/ofp-sam-skj-tag-growth/blob/main/bootstrap/initial/software}%
  {Link to GitHub}} (requires GitHub login).

\section*{Background and model variations}

\subsection*{GCM}

SPC is exploring various methods to estimate growth parameters for the 2022
stock assessment of skipjack tuna. As part of this exploration, there was
interest in fitting a growth cessation model (GCM), described in
\textcolor[rgb]{0,0.1,0.5}%
{\href{https://doi.org/10.1007/s00227-018-3336-9}{Maunder et al. (2018)}}.

Mark Maunder shared the GCM model code with the SPC growth modelling team via
email (2022-05-28). This model is written in Template Model Builder (TMB) and
estimates ages from tagging data, based on the observed length increase between
the date of release and date of recapture.

\begin{tabular}{ll}
  $L_\infty$, $k$, $r_\mathrm{max}$ & growth curve coefficients\\[1ex]
  $A_\mathrm{fix}$, $L_\mathrm{fix}$ & additional growth curve
                                       coefficients\\[1ex]
  $\sigma_a$, $\sigma_b$, $\sigma_\mathrm{MEb}$ & sd(length) coefficients\\[1ex]
  age & vector of estimated age at release for all tagged fish\\[1ex]
\end{tabular}

\subsection*{VBtag}

The SPC data do not show clear signs of growth cessation, so the SPC team
decided to write a similar model that uses a traditional von Bertalanffy model.

The VBtag model is simpler than the GCM model, describing the growth curve with
3 parameters ($L_\infty$, $k$, $t_0$) instead of 5 parameters ($L_\infty$, $k$,
$r_\mathrm{max}$, $A_\mathrm{fix}$, $L_\mathrm{fix}$).

The main difference between VBtag and a `plain vanilla' von Bertalanffy model is
that VBtag estimates the release age of tagged fish, using the same approach as
the GCM model. The parameter vector of estimated ages uses half of the degrees
of freedom from the tagging data, where each tagged fish provides two observed
values to be fitted by the model: length at release and length at recovery.

\subsection*{GCM\_oto}

The GCM\_oto model is the same as the GCM model, with the addition of including
the otolith data. This introduces no additional parameters.

\subsection*{VBtag\_oto}

The VBtag\_oto model is the same as the VBtag model, with the addition of
including the otolith data. This introduces no additional parameters.

\end{document}
